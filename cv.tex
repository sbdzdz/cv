%%%%%%%%%%%%%%%%%%%%%%%%%%%%%%%%%%%%%%%%%
% Friggeri Resume/CV
% XeLaTeX Template
% Version 1.2 (3/5/15)
%
% This template has been downloaded from:
% http://www.LaTeXTemplates.com
%
% Original author:
% Adrien Friggeri (adrien@friggeri.net)
% https://github.com/afriggeri/CV
%
% This version has been modified by:
% Sebastian Dziadzio (sebastian.dziadzio@gmail.com)
%
% License:
% CC BY-NC-SA 3.0 (http://creativecommons.org/licenses/by-nc-sa/3.0/)
%%%%%%%%%%%%%%%%%%%%%%%%%%%%%%%%%%%%%%%%%

\PassOptionsToPackage{dvipsnames,svgnames}{xcolor}
\documentclass[]{friggeri-cv_osx}
\usepackage{fontspec}
\usepackage{fontawesome}
\usepackage[]{hyperref}
\hypersetup{
    colorlinks=false,
    urlcolor=MidnightBlue,
}

\begin{document}

\header{Sebastian }{Dziadzio}{}
\vspace{5mm}
\begin{center}
\href{mailto:sebastian.dziadzio@gmail.com}{\color{gray} \Large \faEnvelope} \hspace{0.05cm}
\href{https://sebastiandziadzio.com}{\color{gray} \Large \faHome} \hspace{0.05cm}
\href{https://github.com/sebastiandziadzio}{\color{gray} \Large \faGithub} \hspace{0.05cm}
\href{https://twitter.com/sbdzdz}{\color{gray} \Large \faTwitter} \hspace{0.05cm}
\href{https://pl.linkedin.com/in/sebastiandziadzio}{\color{gray} \Large \faLinkedin} \hspace{0.05cm}
\href{https://scholar.google.com/citations?user=8vAIQXoAAAAJ\&hl=en}{\color{gray} \Large \faGraduationCap} \hspace{0.05cm}
\end{center}
\vspace{2mm}

\section{Experience}
\begin{entrylist}
\entry
{2022–Now}
{Doctoral Researcher, Bethge Lab, University of Tübingen}
{Tübingen, Germany}
{I am a PhD researcher at the International Max Planck Research School for Intelligent Systems, working with Matthias Bethge in the ELLIS program. My main research interest is post-training of large language models and continual multimodal learning \& evaluation at scale. Anticipated graduation date: 11/2026.\\}

\entry
{2019–2022}
{Scientist, Microsoft Research, Mixed Reality \& AI Lab}
{Cambridge, United Kingdom}
{Developed a parametric 3D face model used across multiple Mixed Reality \& AI projects. I built the training, evaluation, and visualization infrastructure, prioritised and implemented new features, and led a research effort improving model fitting accuracy that reduced the average error by half. I~worked with PyTorch, TensorFlow, and NumPy.\\}

\entry
{2018–2019}
{AI Resident, Microsoft Research}
{Cambridge, United Kingdom}
{I conducted research on deep generative models for novel view synthesis and on deep reinforcement learning for video games. Both projects involved formulating research questions, implementing and extending existing methods, and presenting the results at internal conferences. I used PyTorch and OpenCV.\\}
\entry
{2016–2018}
{Data Scientist, Cliqz}
{Munich, Germany}
{I~designed and implemented an information retrieval model combining LSTM networks and CNNs. I~built a pipeline for
collecting, processing, and visualizing telemetry data.\\}

\entry
{2015–2016}
{Software Engineer, Nokia Networks}
{Kraków, Poland}
{I was responsible for implementing new features for the LTE control plane, as well as designing unit tests, system component tests, and integration tests.\\}
\end{entrylist}


\section{Education}
\begin{entrylist}
\entry
{2014–2016}
{MSc, Computer Science, AGH University}
{Kraków, Poland}
{Final grade: 5/5\\
Selected courses: machine learning, natural language processing, advanced algorithms and data structures, robotics, knowledge representation and reasoning.\\}

\entry
{2010–2014}
{BEng, Acoustical Engineering, AGH University}
{Kraków, Poland}
{Final grade: 4.5/5\\
Selected courses: algebra, calculus, physics, digital signal processing, speech technology, object-oriented system design, image processing, cognitive robotics.}
\end{entrylist}

\newpage
\section{Selected publications}

\begin{entrylistleft}
\entryleft{\href{https://scholar.google.com/citations?user=8vAIQXoAAAAJ\&hl=en}{ONEBench to Test Them All: Sample-Level Benchmarking Over Open-Ended Capabilities}}{2025. A.~Ghosh*, S.~Dziadzio*, A.~Prabhu, V.~Udandarao, S.~Albanie, M.~Bethge.\\
\textit{ACL 2025.}}

\entryleft{\href{https://scholar.google.com/citations?user=8vAIQXoAAAAJ\&hl=en}{How to Merge Your Multimodal Models Over Time?}}{S.~Dziadzio*, V.~Udandarao*, K.~Roth*, A.~Prabhu, Z.~Akata, S.~Albanie, M.~Bethge.\\ \textit{CVPR 2025.}}

\entryleft{\href{https://scholar.google.com/citations?user=8vAIQXoAAAAJ\&hl=en}{A Practitioner's Guide to Continual Multimodal Pretraining}}{2024. K.~Roth*, V.~Udandarao*, S.~Dziadzio, A.~Prabhu, M.~Cherti, O.~Vinyals, O.~Hénaff, S.~Albanie,\\ M.~Bethge, Z.~Akata.\\ \textit{NeurIPS 2024.}}

\entryleft{\href{https://scholar.google.com/citations?user=8vAIQXoAAAAJ\&hl=en}{Disentangled Continual Learning: Separating Memory Edits from Generalization}}{S.~Dziadzio, Ç.~Yıldız, G. v/d Ven, T.~Trzciński, T.~Tuytelaars, M.~Bethge.\\ \textit{CoLLAs 2024.}}

\entryleft{\href{https://scholar.google.com/citations?user=8vAIQXoAAAAJ\&hl=en}{Controllable Image Generation}}{2023. M.~Kowalski, S.~Garbin, M.~Johnson, T.~Baltru\v{s}aitis, M.~De La Gorce, V.~Estellers, S.~Dziadzio.\\
\textit{US Patent US-11748932 2023.}}

\entryleft{\href{https://scholar.google.com/citations?user=8vAIQXoAAAAJ\&hl=en}{Computing Photorealistic Versions of Synthetic Images}}{2022. S.~Garbin, M.~Kowalski, M.~Johnson, T.~Baltru\v{s}aitis, M.~De La Gorce, V.~Estellers, S.~Dziadzio,\\ J.~Shotton.\\
\textit{US Patent US-11354846 2022.}}

\entryleft{\href{https://scholar.google.com/citations?user=8vAIQXoAAAAJ\&hl=en}{Full-Body Motion from a Single Head-Mounted Device}}{2021. A.~Dittadi, S.~Dziadzio, D.~Cosker, B.~Lundell, T.~Cashman, J.~Shotton.\\
\textit{ICCV 2021.}}

\entryleft{\href{https://scholar.google.com/citations?user=8vAIQXoAAAAJ\&hl=en}{Fake It Till You Make It: Face Analysis in the Wild Using Synthetic Data Alone}}{T.~Baltru\v{s}aitis, E.~Wood, C.~Hewitt, S.~Dziadzio, T.~Cashman, J.~Shotton.\\
\textit{ICCV 2021.}}
\end{entrylistleft}

\section{Expertise}
\begin{entrylistleft}
\entryleft{Research areas}{LLM post-training, continual multimodal learning and evaluation, model merging, benchmark design\\}
\entryleft{Technical skills}{Python (PyTorch, TensorFlow, NumPy), distributed training, containerization, inference optimization\\}
\entryleft{Professional skills}{scientific writing, technical project scoping, cross-functional collaboration, agile project management\\}
\entryleft{Languages}{Polish (native), English (fluent), German (intermediate), Spanish (conversational)\\}
\end{entrylistleft}

\end{document}
