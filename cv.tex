%%%%%%%%%%%%%%%%%%%%%%%%%%%%%%%%%%%%%%%%%
% Friggeri Resume/CV
% XeLaTeX Template
% Version 1.2 (3/5/15)
%
% This template has been downloaded from:
% http://www.LaTeXTemplates.com
%
% Original author:
% Adrien Friggeri (adrien@friggeri.net)
% https://github.com/afriggeri/CV
%
% This version has been modified by:
% Sebastian Dziadzio (sebastian.dziadzio@gmail.com)
%
% License:
% CC BY-NC-SA 3.0 (http://creativecommons.org/licenses/by-nc-sa/3.0/)
%%%%%%%%%%%%%%%%%%%%%%%%%%%%%%%%%%%%%%%%%

\PassOptionsToPackage{dvipsnames,svgnames}{xcolor}
\documentclass[]{friggeri-cv_osx}
\usepackage{fontspec}
\usepackage{fontawesome}
\usepackage[]{hyperref}
\hypersetup{
    colorlinks=false,
    urlcolor=MidnightBlue,
}

\begin{document}

\header{Sebastian }{Dziadzio}{}
\vspace{5mm}
\begin{center}
\href{https://sebastiandziadzio.com}{\color{gray} \Large \faHome} \hspace{0.05cm}
\href{https://github.com/sebastiandziadzio}{\color{gray} \Large \faGithub} \hspace{0.05cm}
\href{https://twitter.com/sbdzdz}{\color{gray} \Large\faTwitter} \hspace{0.05cm}
\href{https://pl.linkedin.com/in/sebastiandziadzio}{\color{gray} \Large\faLinkedin} \hspace{0.05cm}
\href{https://scholar.google.com/citations?user=8vAIQXoAAAAJ&hl=en}{\color{gray} \Large\faGraduationCap} \hspace{0.05cm}
\end{center}
\vspace{2mm}

\section{Experience}
\begin{entrylist}
\entry
{2019–2021}
{Scientist, Microsoft Research, Mixed Reality \& AI Lab}
{Cambridge, United Kingdom}
{I worked on a parametric 3D face model that powered multiple projects across the team. I built the training, evaluation, and visualization infrastructure, developed new features, fixed bugs,
and lead a research effort that reduced the average fit error by half. I~worked with PyTorch, TensorFlow, and NumPy.\\}

\entry
{2018–2019}
{AI Resident, Microsoft Research}
{Cambridge, United Kingdom}
{I worked on deep generative models for novel view synthesis and on deep reinforcement learning for video games. Both projects involved
formulating research questions, implementing and extending existing methods, and presenting the results at internal conferences. I used Pytorch, C\#, and OpenCV.\\}

\entry
{2016–2018}
{Data Scientist, Cliqz}
{Munich, Germany}
{I~implemented an information retrieval model combining LSTM networks and CNNs. I~also built a pipeline for
collecting, processing, and visualizing telemetry data. I worked with Python, TensorFlow, and Spark.\\}

\entry
{2015–2016}
{Software Engineer, Nokia Networks}
{Kraków, Poland}
{I was responsible for implementing new features for the LTE control plane, as well as designing unit tests, system component tests,
and integration tests. I~used C++ and Python.\\}

\end{entrylist}


\section{Education}
\begin{entrylist}

\entry
{2022–2025}
{PhD, Computer Science, University of Tübingen}
{Tübingen, Germany}
{I am an ELLIS PhD student working with Matthias Bethge and Tinne Tuytelaars on invariant and equivariant representation learning, disentanglement, and continual learning.\\}

\entry
{2014–2016}
{MSc, Computer Science, AGH University}
{Kraków, Poland}
{Final grade: 5/5\\
Selected courses: machine learning, natural language processing, advanced algorithms and data structures, robotics, knowledge representation and reasoning.\\}

\entry
{2010–2014}
{BEng, Acoustical Engineering, AGH University}
{Kraków, Poland}
{Final grade: 4.5/5\\
Selected courses: algebra, calculus, physics, digital signal processing, speech technology, object-oriented system design, image processing, cognitive robotics.}
\end{entrylist}


\newpage
\section{Publications}
\begin{entrylist}
\entry
{2022}
{\href{https://scholar.google.com/citations?user=8vAIQXoAAAAJ&hl=en}{Computing photorealistic versions of synthetic images}}
{}
{Stephan Garbin, Marek Kowalski, Matthew Johson, Tadas Baltru\v{s}aitis, Martin De La Gorce, Virginia Estellers, Sebastian Dziadzio, Jamie Shotton,
\textit{US Patent US-11354846.}\\}

\entry
{2021}
{\href{https://scholar.google.com/citations?user=8vAIQXoAAAAJ&hl=en}{Fake it Till You Make It: Face Analysis in the Wild Using Synthetic Data Alone}}
{}
{Tadas Baltru\v{s}aitis, Erroll Wood, Charlie Hewitt, Sebastian Dziadzio, Tom Cashman, Jamie Shotton,
\textit{Proceedings of the IEEE/CVF International Conference on Computer Vision.}\\}

\entry
{2021}
{\href{https://scholar.google.com/citations?user=8vAIQXoAAAAJ&hl=en}{Full-Body Motion from a Single Head-Mounted Device: Generating SMPL Poses from Partial Observations}}
{}
{Andrea Dittadi, Sebastian Dziadzio, Darren Cosker, Ben Lundell, Tom Cashman, Jamie Shotton,
\textit{Proceedings of the IEEE/CVF International Conference on Computer Vision.}\\}

\entry
{2021}
{\href{https://scholar.google.com/citations?user=8vAIQXoAAAAJ&hl=en}{Controllable Image Generation}}
{}
{Marek Kowalski, Stephan Garbin, Matthew Johson, Tadas Baltru\v{s}aitis, Martin De La Gorce, Virginia Estellers, Sebastian Dziadzio,
\textit{US Patent Application 16915863.}\\}

\entry
{2015}
{\href{https://scholar.google.com/citations?user=8vAIQXoAAAAJ&hl=en}{Understanding Context with ContextViewer – Tool for Visualization and Preprocessing of Mobile Sensors Data}}
{}
{Szymon Bobek, Sebastian Dziadzio, Paweł Jaciów, Mateusz Ślażyński, Grzegorz J. Nalepa,
\textit{International and Interdisciplinary Conference on Modeling and Using Context.}\\}
\end{entrylist}

\section{Expertise}
\begin{entrylist}
\entry
{}
{Programming}
{}
{Python, PyTorch, TensorFlow, NumPy, C++}

\entry
{}
{Knowledge}
{}
{machine learning, computer vision, natural language processing, software development}

\entry
{}
{Skills}
{}
{presentation, technical and scientific writing, project management}

\entry
{}
{Languages}
{}
{Polish (native), English (fluent), Spanish (conversational), German (conversational)}
\end{entrylist}

\section{Training}
\begin{entrylist}

\entry
{}
{ELLIS Doctoral Symposium 2022}
{University of Alicante}
{1-week workshop}

\entry
{}
{ELLIS Doctoral Symposium 2021}
{University of Tübingen}
{1-week workshop}

\entry
{}
{Oxford Machine Learning Summer School 2021 \& 2020}
{Oxford University}
{1-week online workshop}

\entry
{}
{AI Residency Bootcamp}
{Microsoft Research}
{2-week workshop followed by a series of lectures}

\entry
{}
{Deep Learning Specialization}
{Coursera}
{20-week online course, 4 hours per week}

\entry
{}
{Machine Learning}
{Coursera}
{11-week online course, 8 hours per week}

\end{entrylist}
\end{document}
