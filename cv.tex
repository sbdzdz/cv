%%%%%%%%%%%%%%%%%%%%%%%%%%%%%%%%%%%%%%%%%
% Friggeri Resume/CV
% XeLaTeX Template
% Version 1.2 (3/5/15)
%
% This template has been downloaded from:
% http://www.LaTeXTemplates.com
%
% Original author:
% Adrien Friggeri (adrien@friggeri.net)
% https://github.com/afriggeri/CV
%
% This version has been modified by:
% Sebastian Dziadzio (sebastian.dziadzio@gmail.com) 
%
% License:
% CC BY-NC-SA 3.0 (http://creativecommons.org/licenses/by-nc-sa/3.0/)
%%%%%%%%%%%%%%%%%%%%%%%%%%%%%%%%%%%%%%%%%

\documentclass[]{friggeri-cv_osx}
\usepackage{fontspec}
\usepackage{fontawesome}

\begin{document}

\header{Sebastian }{Dziadzio}{}
\vspace{5mm}
\begin{center}
\href{https://sebastiandziadzio.com}{\color{gray} \Large \faHome} \hspace{0.05cm}
\href{https://github.com/sebastiandziadzio}{\color{gray} \Large \faGithub} \hspace{0.05cm}
\href{mailto:sebastian.dziadzio@gmail.com}{\color{gray} \Large\faEnvelope} \hspace{0.05cm}
\href{https://twitter.com/sebadzia}{\color{gray} \Large\faTwitter} \hspace{0.05cm}
\href{https://pl.linkedin.com/in/sebastiandziadzio}{\color{gray} \Large\faLinkedin} \hspace{0.05cm}
\end{center}
%%\vspace{2mm}

\section{Experience}
\begin{entrylist}
\entry
{2019-Now}
{Scientist, Microsoft Cognition}
{Cambridge, United Kingdom}
{I work on computer vision and graphics, in particular deep generative models and learned parameterized 3D models.\\}

\entry
{2018-2019}
{AI Resident, Microsoft Research}
{Cambridge, United Kingdom}
{I worked on deep reinforcement learning for games and deep generative models for novel view synthesis. Both projects required
reviewing literature, choosing and implementing models, designing and conducting experiments, and extending existing methods.
I presented my results at internal conferences and events.\\}

\entry
{2016-2018}
{Software Engineer, Cliqz}
{Munich, Germany}
{I built a deep information retrieval model based on a research paper. Previously, I designed and implemented a data pipeline for
collecting, processing, analyzing, and visualizing telemetry data.\\}

\entry
{2015-2016}
{Software Engineer, Nokia Networks}
{Kraków, Poland}
{I was responsible for implementing new features for LTE base transceiver stations, as well as designing unit tests, system component tests,
and integration tests.\\}

\entry
{2014-2015}
{Research Intern, AGH Signal Processing Group}
{Kraków, Poland}
{I investigated the use of language models in interactive voice response systems and conducted a comparative analysis of language models
in automatic speech recognition. I published my results in a conference paper.\\}
\end{entrylist}


\section{Education}
\begin{entrylist}
\entry
{2014-2016}
{M.Sc. in Computer Science, Intelligent Systems}
{AGH University}
{Final grade: 5/5\\
Thesis: Application of Morphosyntactic and Semantic Language Models in Automatic Speech Recognition.\\
Key courses: natural language processing, advanced algorithms, machine learning, multimodal interfaces, computational intelligence.}

\entry
{2010-2014}
{B.Eng. in Acoustical Engineering, Vibroacoustics}
{AGH University}
{Final grade: 4.5/5\\
Thesis: Unit Selection Text to Speech System for Polish.\\
Selected courses: algebra, calculus, speech technology, physics, object-oriented system design, image processing, cognitive robotics.}
\end{entrylist}


\newpage
\section{Publications}
\begin{entrylist}
\entry
{2015}
{Comparing Language Models Trained on Written Texts and Speech Transcripts\\}
{S. Dziadzio, A. Nabożny, A. Pohl, B. Ziółko}
{Proceedings of the 10\textsuperscript{th} International Symposium on Advances in Artificial Intelligence and Applications\\}

\entry
{2015}
{Understanding Context with ContextViewer – Tool for Visualization and Initial Preprocessing of Mobile Sensors Data\\}
{S. Bobek, S. Dziadzio, P. Jaciów, M. Ślażyński, G. Nalepa}
{Proceedings of the 9\textsuperscript{th} International and Interdisciplinary Conference on Modeling and Using Context}
\end{entrylist}

\section{Expertise}
\begin{entrylist}
\entry
{}
{Programming}
{}
{Python, NumPy, OpenCV, PyTorch, TensorFlow, C++}

\entry
{}
{Knowledge}
{}
{machine learning, computer vision, natural language processing, software development}

\entry
{}
{Skills}
{}
{presentation, scientific writing, implementing ideas from research papers}

\entry
{}
{Languages}
{}
{Polish (native), English (fluent), Spanish (conversational), German (conversational)}
\end{entrylist}

\section{Training}
\begin{entrylist}
\entry
{}
{AI Residency Bootcamp}
{Microsoft Research}
{2-week workshop followed by a series of lectures}

\entry
{}
{Programming and Machine Learning}
{Cliqz}
{2-week workshop}

\entry
{}
{Advanced Python}
{Nokia}
{3-day workshop}

\entry
{}
{C++ STL}
{Nokia}
{3-day workshop}

\entry
{}
{C++ Programming}
{Nokia}
{4-week course, 16 hours per week}

\entry
{}
{Deep Learning}
{Coursera}
{20-week online course, 4 hours per week}

\entry
{}
{Machine Learning}
{Coursera}
{11-week online course, 8 hours per week}

\entry
{}
{Learning from Data}
{CaltechX}
{10-week online course, 20 hours per week}

\entry
{}
{Artificial Intelligence}
{BerkeleyX}
{12-week online course, 10 hours per week}
\end{entrylist}
\end{document}
