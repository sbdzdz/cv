%%%%%%%%%%%%%%%%%%%%%%%%%%%%%%%%%%%%%%%%%
% Friggeri Resume/CV
% XeLaTeX Template
% Version 1.2 (3/5/15)
%
% This template has been downloaded from:
% http://www.LaTeXTemplates.com
%
% Original author:
% Adrien Friggeri (adrien@friggeri.net)
% https://github.com/afriggeri/CV
%
% This version has been modified by:
% Sebastian Dziadzio (sebastian.dziadzio@gmail.com)
%
% License:
% CC BY-NC-SA 3.0 (http://creativecommons.org/licenses/by-nc-sa/3.0/)
%%%%%%%%%%%%%%%%%%%%%%%%%%%%%%%%%%%%%%%%%

\PassOptionsToPackage{dvipsnames,svgnames}{xcolor}
\documentclass[]{friggeri-cv_osx}
\usepackage{fontspec}
\usepackage{fontawesome}
\usepackage[]{hyperref}
\hypersetup{
    colorlinks=false,
    urlcolor=MidnightBlue,
}

\begin{document}

\header{Sebastian }{Dziadzio}{}
\vspace{5mm}
\begin{center}
\href{mailto:sebastian.dziadzio@gmail.com}{\color{gray} \Large \faEnvelope} \hspace{0.05cm}
\href{https://sebastiandziadzio.com}{\color{gray} \Large \faHome} \hspace{0.05cm}
\href{https://github.com/sebastiandziadzio}{\color{gray} \Large \faGithub} \hspace{0.05cm}
\href{https://twitter.com/sbdzdz}{\color{gray} \Large \faTwitter} \hspace{0.05cm}
\href{https://pl.linkedin.com/in/sebastiandziadzio}{\color{gray} \Large \faLinkedin} \hspace{0.05cm}
\href{https://scholar.google.com/citations?user=8vAIQXoAAAAJ&hl=en}{\color{gray} \Large \faGraduationCap} \hspace{0.05cm}
\end{center}
\vspace{2mm}

\section{Experience}
\begin{entrylist}
\entry
{2022–Now}
{Doctoral Researcher, Bethge Lab, University of Tübingen}
{Tübingen, Germany}
{I am a PhD researcher at the International Max Planck Research School for Intelligent Systems, working with Matthias Bethge in the ELLIS program. My main research interest is post-training of large language models and continual multimodal learning \& evaluation at scale. Anticipated graduation date: 11/2026.\\}

\entry
{2019–2022}
{Scientist, Microsoft Research, Mixed Reality \& AI Lab}
{Cambridge, United Kingdom}
{I worked on a parametric 3D face model used across multiple Mixed Reality \& AI projects. I built the training, evaluation, and visualization infrastructure, developed new features, fixed bugs, and led a research effort improving model fitting accuracy that reduced the average error by half. I~worked with PyTorch, TensorFlow, and NumPy.\\}

\entry
{2018–2019}
{AI Resident, Microsoft Research}
{Cambridge, United Kingdom}
{I conducted research on deep generative models for novel view synthesis and on deep reinforcement learning for video games. Both projects involved formulating research questions, implementing and extending existing methods, and presenting the results at internal conferences. I used PyTorch and OpenCV.\\}
\entry
{2016–2018}
{Data Scientist, Cliqz}
{Munich, Germany}
{I~designed and implemented an information retrieval model combining LSTM networks and CNNs. I~built a pipeline for
collecting, processing, and visualizing telemetry data.\\}

\entry
{2015–2016}
{Software Engineer, Nokia Networks}
{Kraków, Poland}
{I was responsible for implementing new features for the LTE control plane, as well as designing unit tests, system component tests, and integration tests.\\}
\end{entrylist}


\section{Education}
\begin{entrylist}
\entry
{2014–2016}
{MSc, Computer Science, AGH University}
{Kraków, Poland}
{Final grade: 5/5\\
Selected courses: machine learning, natural language processing, advanced algorithms and data structures, robotics, knowledge representation and reasoning.\\}

\entry
{2010–2014}
{BEng, Acoustical Engineering, AGH University}
{Kraków, Poland}
{Final grade: 4.5/5\\
Selected courses: algebra, calculus, physics, digital signal processing, speech technology, object-oriented system design, image processing, cognitive robotics.}
\end{entrylist}

\newpage
\section{Selected publications}

\begin{entrylist}
\entry
{2025}
{ONEBench: Sample-Level Benchmarking Over Open-Ended Capabilities}
{}
{Adhiraj Ghosh*, Sebastian Dziadzio*, Ameya Prabhu, Vishal Udandarao, Samuel Albanie, Matthias Bethge.
\textit{ACL 2025.}\\}

\entry
{}
{How to Merge Your Multimodal Models Over Time?}
{}
{Sebastian Dziadzio*, Vishaal Udandarao*, Karsten Roth*, Ameya Prabhu, Zeynep Akata, Samuel Albanie, Matthias Bethge.
\textit{CVPR 2025.}\\}

\entry
{2024}
{A Practitioner's Guide to Continual Multimodal Pretraining}
{}
{Karsten Roth*, Vishaal Udandarao*, Sebastian Dziadzio, Ameya Prabhu, Mehdi Cherti, Oriol Vinyals, Olivier Hénaff, Samuel Albanie, Matthias Bethge, Zeynep Akata.
\textit{NeurIPS 2024.}\\}

\entry
{}
{\href{https://scholar.google.com/citations?user=8vAIQXoAAAAJ&hl=en}{Infinite dSprites for Disentangled Continual Learning: Separating Memory Edits from Generalization}}
{}
{Sebastian Dziadzio, Çağatay Yıldız, Gido M. van de Ven, Tomasz Trzciński, Tinne Tuytelaars, Matthias Bethge.
\textit{CoLLAs 2024.}\\}

\entry
{2023}
{\href{https://scholar.google.com/citations?user=8vAIQXoAAAAJ&hl=en}{Controllable Image Generation}}
{}
{Marek Kowalski, Stephan Garbin, Matthew Johson, Tadas Baltru\v{s}aitis, Martin De La Gorce, Virginia Estellers, Sebastian Dziadzio.
\textit{US Patent US-11748932.}\\}

\entry
{2022}
{\href{https://scholar.google.com/citations?user=8vAIQXoAAAAJ&hl=en}{Computing Photorealistic Versions of Synthetic Images}}
{}
{Stephan Garbin, Marek Kowalski, Matthew Johson, Tadas Baltru\v{s}aitis, Martin De La Gorce, Virginia Estellers, Sebastian Dziadzio, Jamie Shotton.
\textit{US Patent US-11354846.}\\}

\entry
{2021}
{\href{https://scholar.google.com/citations?user=8vAIQXoAAAAJ&hl=en}{Full-Body Motion from a Single Head-Mounted Device: Generating SMPL Poses from Partial Observations}}
{}
{Andrea Dittadi, Sebastian Dziadzio, Darren Cosker, Ben Lundell, Tom Cashman, Jamie Shotton.
\textit{ICCV 2021.}\\}

\entry
{}
{\href{https://scholar.google.com/citations?user=8vAIQXoAAAAJ&hl=en}{Fake It Till You Make It: Face Analysis in the Wild Using Synthetic Data Alone}}
{}
{Tadas Baltru\v{s}aitis, Erroll Wood, Charlie Hewitt, Sebastian Dziadzio, Tom Cashman, Jamie Shotton.
\textit{ICCV 2021.}\\}
\end{entrylist}

\section{Expertise}
\begin{entrylist}
\entry
{}
{Programming}
{}
{Python (PyTorch, TensorFlow, NumPy), C++}

\entry
{}
{Knowledge}
{}
{machine learning, continual multimodal learning, software development}

\entry
{}
{Skills}
{}
{public speaking, technical and scientific writing, agile project management}

\entry
{}
{Languages}
{}
{Polish (native), English (fluent), German (intermediate), Spanish (conversational)}
\end{entrylist}
\end{document}
