%%%%%%%%%%%%%%%%%%%%%%%%%%%%%%%%%%%%%%%%%
% Friggeri Resume/CV
% XeLaTeX Template
% Version 1.2 (3/5/15)
%
% This template has been downloaded from:
% http://www.LaTeXTemplates.com
%
% Original author:
% Adrien Friggeri (adrien@friggeri.net)
% https://github.com/afriggeri/CV
%
% This version has been modified by:
% Sebastian Dziadzio (sebastian.dziadzio@gmail.com)
%
% License:
% CC BY-NC-SA 3.0 (http://creativecommons.org/licenses/by-nc-sa/3.0/)
%%%%%%%%%%%%%%%%%%%%%%%%%%%%%%%%%%%%%%%%%

\PassOptionsToPackage{dvipsnames,svgnames}{xcolor}
\documentclass[]{friggeri-cv_osx}
\usepackage{fontspec}
\usepackage{fontawesome}
\usepackage[]{hyperref}
\hypersetup{
    colorlinks=true,
    urlcolor=MidnightBlue,
}

\begin{document}

\header{Sebastian }{Dziadzio}{}
\vspace{5mm}
\begin{center}
\href{https://sebastiandziadzio.com}{\color{gray} \Large \faHome} \hspace{0.05cm}
\href{https://github.com/sebastiandziadzio}{\color{gray} \Large \faGithub} \hspace{0.05cm}
\href{mailto:sebastian.dziadzio@gmail.com}{\color{gray} \Large\faEnvelope} \hspace{0.05cm}
\href{https://twitter.com/sebadzia}{\color{gray} \Large\faTwitter} \hspace{0.05cm}
\href{https://pl.linkedin.com/in/sebastiandziadzio}{\color{gray} \Large\faLinkedin} \hspace{0.05cm}
\end{center}
\vspace{2mm}

\section{Experience}
\begin{entrylist}
\entry
{2019-Now}
{Scientist, Microsoft Mixed Reality \& AI Lab}
{Cambridge, United Kingdom}
{I worked on a parametric 3D face model that powered a number of projects across the team. I built a training and evaluation infrastructure
that helped double the accuracy of the model. I worked with PyTorch, TensorFlow, and NumPy.\\}

\entry
{2018-2019}
{AI Resident, Microsoft Research}
{Cambridge, United Kingdom}
{I worked on deep reinforcement learning for games and deep generative models for novel view synthesis. Both projects required
formulating research questions, implementing and extending existing methods, and presenting results at internal conferences.\\}

\entry
{2016-2018}
{Software Engineer, Cliqz}
{Munich, Germany}
{I implemented an information retrieval model combining LSTM and convolutional networks. I also built a pipeline for
collecting, processing, and visualizing telemetry data.\\}

\entry
{2015-2016}
{Software Engineer, Nokia Networks}
{Kraków, Poland}
{I was responsible for implementing new features for the LTE control plane, as well as designing unit tests, system component tests,
and integration tests.\\}

\entry
{2014-2015}
{Research Intern, AGH Signal Processing Group}
{Kraków, Poland}
{I conducted a comparative analysis of language models in automatic speech recognition and published results in a conference paper.\\}
\end{entrylist}


\section{Education}
\begin{entrylist}

\entry
{2022-2025}
{PhD, Computer Science}
{University of Tübingen}
{I work with Matthias Bethge and Tinne Tuytelaars on unsupervised continual learning in computer vision.\\}

\entry
{2014-2016}
{MSc, Computer Science, Intelligent Systems}
{AGH University}
{Final grade: 5/5\\
Thesis: Application of Morphosyntactic and Semantic Language Models in Automatic Speech Recognition.\\
Selected courses: machine learning, natural language processing, advanced algorithms and data structures, robotics, knowledge representation and reasoning.\\}

\entry
{2010-2014}
{BEng, Acoustical Engineering}
{AGH University}
{Final grade: 4.5/5\\
Thesis: Unit Selection Text to Speech System for Polish.\\
Selected courses: algebra, calculus, physics, digital signal processing, speech technology, object-oriented system design, image processing, cognitive robotics.}
\end{entrylist}


\newpage
\section{Publications}
\begin{entrylist}
\entry
{2021}
{\href{https://scholar.google.com/citations?user=8vAIQXoAAAAJ&hl=en}{Fake it Till You Make It: Face Analysis in the Wild Using Synthetic Data Alone\\}}
{Tadas Baltrusaitis, Erroll Wood, Charlie Hewitt, Sebastian Dziadzio, Tom Cashman, Jamie Shotton}
{Proceedings of the IEEE/CVF International Conference on Computer Vision\\}

\entry
{2021}
{\href{https://scholar.google.com/citations?user=8vAIQXoAAAAJ&hl=en}{Full-Body Motion from a Single Head-Mounted Device: Generating SMPL Poses from Partial Observations\\}}
{Andrea Dittadi, Sebastian Dziadzio, Darren Cosker, Ben Lundell, Tom Cashman, Jamie Shotton}
{Proceedings of the IEEE/CVF International Conference on Computer Vision\\}

\entry
{2015}
{\href{https://scholar.google.com/citations?user=8vAIQXoAAAAJ&hl=en}{Understanding Context with ContextViewer – Tool for Visualization and Preprocessing of Mobile Sensors Data\\}}
{Szymon Bobek, Sebastian Dziadzio, Paweł Jaciów, Mateusz Ślażyński, Grzegorz J. Nalepa}
{Proceedings of the 9\textsuperscript{th} International and Interdisciplinary Conference on Modeling and Using Context\\}

\entry
{2015}
{\href{https://scholar.google.com/citations?user=8vAIQXoAAAAJ&hl=en}{Comparing Language Models Trained on Written Texts and Speech Transcripts\\}}
{Sebastian Dziadzio, Aleksandra Nabożny, Aleksander Pohl, Bartosz Ziółko}
{Proceedings of the 10\textsuperscript{th} International Symposium on Advances in Artificial Intelligence and Applications\\}
\end{entrylist}

\section{Expertise}
\begin{entrylist}
\entry
{}
{Programming}
{}
{Python, PyTorch, TensorFlow, NumPy, OpenCV, C++}

\entry
{}
{Knowledge}
{}
{machine learning, computer vision, natural language processing, software development}

\entry
{}
{Skills}
{}
{presentation, scientific writing, implementing ideas from research papers}

\entry
{}
{Languages}
{}
{Polish (native), English (fluent), Spanish (conversational), German (conversational)}
\end{entrylist}

\section{Training}
\begin{entrylist}
\entry
{}
{Oxford Machine Learning Summer School 2021}
{Oxford University}
{1-week workshop}

\entry
{}
{Oxford Machine Learning Summer School 2020}
{Oxford University}
{1-week workshop}

\entry
{}
{AI Residency Bootcamp}
{Microsoft Research}
{2-week workshop followed by a series of lectures}

\entry
{}
{Programming and Machine Learning}
{Cliqz}
{2-week workshop}

\entry
{}
{Advanced Python}
{Nokia}
{3-day workshop}

\entry
{}
{Deep Learning Specialization}
{Coursera}
{20-week online course, 4 hours per week}

\entry
{}
{Machine Learning}
{Coursera}
{11-week online course, 8 hours per week}

\entry
{}
{Learning from Data}
{CaltechX}
{10-week online course, 20 hours per week}

\end{entrylist}
\end{document}
