%%%%%%%%%%%%%%%%%%%%%%%%%%%%%%%%%%%%%%%%%
% Friggeri Resume/CV
% XeLaTeX Template
% Version 1.2 (3/5/15)
%
% This template has been downloaded from:
% http://www.LaTeXTemplates.com
%
% Original author:
% Adrien Friggeri (adrien@friggeri.net)
% https://github.com/afriggeri/CV
%
% License:
% CC BY-NC-SA 3.0 (http://creativecommons.org/licenses/by-nc-sa/3.0/)
%
% Important notes:
% This template needs to be compiled with XeLaTeX and the bibliography, if used,
% needs to be compiled with biber rather than bibtex.
%
%%%%%%%%%%%%%%%%%%%%%%%%%%%%%%%%%%%%%%%%%

\documentclass[]{friggeri-cv} % Add 'print' as an option into the square bracket to remove colors from this template for printing

\addbibresource{bibliography.bib} % Specify the bibliography file to include publications
\usepackage{fontawesome}

\begin{document}
\header{sebastian}{dziadzio}{software engineer} % Your name and current job title/field

%----------------------------------------------------------------------------------------
%	SIDEBAR SECTION
%----------------------------------------------------------------------------------------

\begin{aside} % In the aside, each new line forces a line break
\section{contact}
+48 608 687 500
\href{mailto:sebastian.dziadzio@gmail.com}{sebastian.dziadzio@gmail.com}
\href{http://pl.linkedin.com/in/sebastiandziadzio}{\color{gray} \faLinkedinSign} \href{https://github.com/sebastiandziadzio}{\color{gray} \faGithubSign} \href{https://twitter.com/sebadzia}{\color{gray} \faTwitterSign}
\section{software}
C++, Python
Ruby, Git, Bash
\section{knowledge}
software design
machine learning
speech recognition
\section{languages}
Polish - native
English - fluent
Spanish - intermediate
German - conversational
\end{aside}

%----------------------------------------------------------------------------------------
%	EDUCATION SECTION
%----------------------------------------------------------------------------------------

\section{education}

\begin{entrylist}

\entry
{2014-2015}
{Master of Computer Science}
{AGH University}
{Intelligent Systems \\ Key courses: advanced algorithms, neural networks, intelligent mobile technology, machine learning, natural language processing.\\}

\entry
{2010-2014}
{Bachelor of Acoustical Engineering}
{AGH University}
{Vibroacoustics \\ Key courses: algebra, calculus, digital signal processing, object-oriented programming, image processing, speech technology.}
\end{entrylist}

%----------------------------------------------------------------------------------------
%	WORK EXPERIENCE SECTION
%----------------------------------------------------------------------------------------

\section{experience}
\begin{entrylist}

\entry
{2015--Now}
{Nokia Networks}
{Cracow, Poland}
{Software Engineer \\
  I am responsible for implementing new features for the LTE control plane~(C++), unit testing (Google Test), component testing~(TTCN-3), and~integration testing (Python). I~work in an agile environment (Scrum).\\}

\entry
{2014--2015}
{AGH University}
{Cracow, Poland}
{Junior Researcher \\
  As a member of a research team, I investigated the use of semantic and syntactic language models in interactive voice response systems and described my results in a conference paper.\\}

\entry
{2013}
{AGH University}
{Cracow, Poland}
{Summer Intern \\
  I conducted a comparative analysis of morphosyntactic language models in the context of automatic speech recognition.}

\end{entrylist}

%----------------------------------------------------------------------------------------
%	AWARDS SECTION
%----------------------------------------------------------------------------------------

\section{awards}

\begin{entrylist}

\entry
{2010--2012}
{Chancellor's Scholarship}
{AGH University}
{Awarded annually based on academic record.}
\end{entrylist}

%----------------------------------------------------------------------------------------
%	PUBLICATIONS SECTION
%----------------------------------------------------------------------------------------

\section{publications}

\printbibsection{inproceedings}{}
\subsubsection{}Oświadczam, że wyrażam zgodę.
\end{document}
